% This is the Reed College LaTeX thesis template. Most of the work 
% for the document class was done by Sam Noble (SN), as well as this
% template. Later comments etc. by Ben Salzberg (BTS). Additional
% restructuring and APA support by Jess Youngberg (JY).
% Your comments and suggestions are more than welcome; please email
% them to cus@reed.edu
%
% See http://web.reed.edu/cis/help/latex.html for help. There are a 
% great bunch of help pages there, with notes on
% getting started, bibtex, etc. Go there and read it if you're not
% already familiar with LaTeX.
%
% Any line that starts with a percent symbol is a comment. 
% They won't show up in the document, and are useful for notes 
% to yourself and explaining commands. 
% Commenting also removes a line from the document; 
% very handy for troubleshooting problems. -BTS

% As far as I know, this follows the requirements laid out in 
% the 2002-2003 Senior Handbook. Ask a librarian to check the 
% document before binding. -SN

%%
%% Preamble
%%
% \documentclass{<something>} must begin each LaTeX document
\documentclass[12pt,twoside]{reedthesis}
% Packages are extensions to the basic LaTeX functions. Whatever you
% want to typeset, there is probably a package out there for it.
% Chemistry (chemtex), screenplays, you name it.
% Check out CTAN to see: http://www.ctan.org/
%%
\usepackage{graphicx,latexsym} 
\usepackage{amssymb,amsthm,amsmath}
\usepackage{longtable,booktabs,setspace} 
\usepackage[hyphens]{url}
\usepackage{rotating}
\usepackage{natbib}
% Comment out the natbib line above and uncomment the following two lines to use the new 
% biblatex-chicago style, for Chicago A. Also make some changes at the end where the 
% bibliography is included. 
%\usepackage{biblatex-chicago}
%\bibliography{thesis}
\usepackage{lipsum}
\usepackage{hyperref}

% \usepackage{times} % other fonts are available like times, bookman, charter, palatino

% Expose metadata as LaTeX commands so that they can be used in the Markdown files that make up the thesis
\newcommand{\commitRev}{$commitRev$}
\newcommand{\commitShortRev}{$commitShortRev$}
\newcommand{\commitDate}{$commitDate$}

% Syntax highlighting
$if(highlighting-macros)$
  $highlighting-macros$
$endif$

% From Pandoc
\providecommand{\tightlist}{%
  \setlength{\itemsep}{0pt}\setlength{\parskip}{0pt}}

\title{$title$}
\author{$author$}
% The month and year that you submit your FINAL draft TO THE LIBRARY (May or December)
\date{$date$}
\division{$division$}
\advisor{$advisor$}
%If you have two advisors for some reason, you can use the following
%\altadvisor{Your Other Advisor}
%%% Remember to use the correct department!
\department{$department$}
% if you're writing a thesis in an interdisciplinary major,
% uncomment the line below and change the text as appropriate.
% check the Senior Handbook if unsure.
%\thedivisionof{The Established Interdisciplinary Committee for}
% if you want the approval page to say "Approved for the Committee",
% uncomment the next line
%\approvedforthe{Committee}

\setlength{\parskip}{0pt}
%%
%% End Preamble
%%
%% The fun begins:
\begin{document}

    % \href{https://github.com/Samasaur1/thesis/commit/\commitRev}{commit \commitShortRev}

    % \fancyfoot[CE,CO]{Built off of commit $rev$ on $date$}
    \fancyfoot[LE]{Built off of \href{https://github.com/Samasaur1/thesis/commit/\commitRev}{commit \commitShortRev}}
    % \fancyfoot[LE]{Built off of commit \commitShortRev}

  \maketitle
  \frontmatter % this stuff will be roman-numbered
  \pagestyle{empty} % this removes page numbers from the frontmatter

% Acknowledgements (Acceptable American spelling) are optional
% So are Acknowledgments (proper English spelling)
    \chapter*{Acknowledgements}
	I want to thank a few people.

% The preface is optional
% To remove it, comment it out or delete it.
    \chapter*{Preface}
	This is an example of a thesis setup to use the reed thesis document class.
	
    % Only add a list of abbreviations if there are any abbreviations
    $if(abbreviations)$
    \chapter*{List of Abbreviations}
    \begin{longtable}{p{.20\textwidth} | p{.80\textwidth}}
      $for(abbreviations/pairs)$
        \textbf{$it.key$} & $it.value$ \\
      $endfor$
    \end{longtable}
    $endif$

    \tableofcontents
% if you want a list of tables, optional
    \listoftables
% if you want a list of figures, also optional
    \listoffigures

% The abstract is not required if you're writing a creative thesis (but aren't they all?)
% If your abstract is longer than a page, there may be a formatting issue.
    \chapter*{Abstract}
	The preface pretty much says it all.
	
	\chapter*{Dedication}
	You can have a dedication here if you wish.

  \mainmatter % here the regular arabic numbering starts
  \pagestyle{fancyplain} % turns page numbering back on

  $body$

% %This is where endnotes are supposed to go, if you have them.
% %I have no idea how endnotes work with LaTeX.
%
%   \backmatter % backmatter makes the index and bibliography appear properly in the t.o.c...
%
% % if you're using bibtex, the next line forces every entry in the bibtex file to be included
% % in your bibliography, regardless of whether or not you've cited it in the thesis.
%     \nocite{*}
%
% % Rename my bibliography to be called "Works Cited" and not "References" or ``Bibliography''
% % \renewcommand{\bibname}{Works Cited}
%
% %    \bibliographystyle{bsts/mla-good} % there are a variety of styles available; 
% %  \bibliographystyle{plainnat}
% % replace ``plainnat'' with the style of choice. You can refer to files in the bsts or APA 
% % subfolder, e.g. 
%  \bibliographystyle{APA/apa-good}  % or
%  \bibliography{thesis}
%  % Comment the above two lines and uncomment the next line to use biblatex-chicago.
%  %\printbibliography[heading=bibintoc]
%
% % Finally, an index would go here... but it is also optional.
\end{document}
